\documentclass[10pt,a4paper]{article}

\usepackage{fullpage}
\usepackage{wrapfig}
\usepackage{lipsum}
\usepackage{hyperref}
\usepackage{cleveref}
\usepackage{tikz}
\usepackage{float}
\usepackage{comment}

\DeclareGraphicsExtensions{.pdf,.png,.jpg}

\begin{document}
\title{WebApp Group 34 Final Report}
\author{
  Han, Qiao\\
  \and
  Chabierski, Piotr\\
  \and
  Smith, Bradley\\
  \and
  Cingillioglu, Nuri\\
}

\maketitle

\section{Introduction}
Our project is a web based application that complements the process of creating and tracking of events and people who are attending the event. It also provides facilities to allow for participants to ask questions and find out information from the event organisers and for event organisers to control groups of people who are able to create events. 
\\
\\
\noindent This application contains several entities:
\begin{description}
\item[Calendars] - these are entities that any user can create. They contain   
\end{description}

\section{Libraries}
Libraries that are used in the front-end:
\begin{description}
\item[jQuery] - We use version 2.1.4 which like any other version is licensed under the MIT license that permits any applicable usage. Although we could, we did not modify any sections of the library. https://jquery.com
\item[jQuery Datetime Picker] - We use this library to bind picking of date time input fields in the forms. Like jQuery the library is under the MIT license. http://xdsoft.net/jqplugins/datetimepicker/
\item[Bootstrap] - Bootstrap is also licensed under MIT. http://getbootstrap.com/
\item[Toastr Js] - MIT license. https://github.com/CodeSeven/toastr
\end{description}
It is important to note that all of the libraries the front-end uses are under the MIT license which means that we have the permission to reuse within proprietary software, for example we do not disclose the source code and charge for the usage, provided that all copies of the licensed software include a copy of the MIT License terms and the copyright notice. \\ \\
Libraries that are used in the back-end:
\begin{description}
\item[Google GSON] - The GSON library is used to serialize and deserialize Java objects into Json strings when they are sent to and recieved from the front-end. It is licensed under the Apache License 2.0. https://code.google.com/p/google-gson/
\item[PostgreSQL JDBC Driver] - The Java back-end is linked to the group database using the driver. Licensed under BSD license. https://jdbc.postgresql.org/
\end{description}
\end{document}
