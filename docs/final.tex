\documentclass[10pt,a4paper]{article}

\usepackage{fullpage}
\usepackage{wrapfig}
\usepackage{lipsum}
\usepackage{hyperref}
\usepackage{cleveref}
\usepackage{tikz}
\usepackage{float}
\usepackage{comment}

\DeclareGraphicsExtensions{.pdf,.png,.jpg}

\begin{document}
\title{WebApp Group 34 Final Report}
\author{
  Han, Qiao\\
  \and
  Chabierski, Piotr\\
  \and
  Smith, Bradley\\
  \and
  Cingillioglu, Nuri\\
}

\maketitle

\section{Introduction}
Our project is a web based application that complements the process of creating 
and tracking of events and people who are attending them. It also provides 
facilities to allow for participants to ask questions and find out information 
from the event organisers and for event organisers to control groups of people 
who are able to create events. 
\\
\\
\noindent This application contains two main entities:
\begin{description}
\item[Calendars] - these are entities that any user can create. User can 
subscribe to other peoples calendars where they can then be promoted to editors 
or admins. Depending on their role they will have different rights and available 
actions is described \textbf{put-link-here}.
\item[Events] - events are entities that are created on the calendar and are 
available to join to anyone who has subscribed. They contain the following 
information: a title, a start date and time, an end end and time, a location, a 
max capacity (possibly unlimited) and a description.
\end{description}

\noindent We decided on creating this application due to a problem that one of 
our team members spotted during volunteering for the collage. This problem was 
with the process of organising events, which typically involves large quantities of 
emails to send out the information and get responses back from interested 
volunteers and either send out more email telling people that events are oversubscribed or 
asking for more people as they didn't get enough.
\\
\\
\noindent Our aims for this project were to improve our understanding of how to 
build products by working with the purposed end users. We also aimed to gain 
greater understanding of web development and database management along with a new 
range of technology in which most of us were novices. We have been working quite 
closely with the users to ensure that the features we have been implementing 
will appeal to them and can be integrated into their current work processes.
\\
\\
\noindent The basic requirements for our project were to make event organising 
easier (to do away with the huge amount of emails). We also gained quite a few 
requirements/possible extensions from conversing with the user.
\\
\\
\noindent These were (from meeting minutes):
\begin{itemize}
\setlength\itemsep{0.1em}
\item Allow event details to be edited after creation.
\item Allow admins to check who has signed up for an event they have created.
\item Allow for multiple admins on the same calendar to collaborate in setting 
up events. 
\item Make it so events can have prerequisites which users must identify they 
have read before they can sign up.
\item Allow admins to remove people from the events if they do not meet the 
prerequisites (or for any other reason).
\item Always start the calendar view on a Monday and highlight the current day.
\item Admin approval for events that were created by editors.
\item In-app live chat and offline messages for volunteers to ask questions and 
get answers.
\item Real time updates for new events and event subscriptions.
\item Make it easy to join and edit events from a mobile browser.  
\end{itemize}
\noindent We also got a lot more suggestions which we have either discarded due 
to coming up with a better alternative or put off in favour of more important 
features. Examples of these are email alerts of upcoming/important events, 
statistics on how much a user has volunteered (what events they have attended), 
report generation for events that have been created for each week, etc...

\section{Project Management}

\noindent The management of our project was realised with daily meetings to 
discuss progress and work out what we needed to work on each day. As we spend 
most of the time working in close proximity with each other it allowed us to 
communicate quickly on all matters and minimize redundant work (writing emails
or other sort of messages). Using Trello board allowed our team to manage
ideas in a consistent manner, set up goals and track our progress effectively. 
We also took advantage of the user meetings (mentioned above) to help keep us on track. 

\subsection{Group Structure}
\noindent We divided our group responsibilities based on our individual 
strengths. Han has done extensive development work on both server and client 
side applications in past internships and part time jobs and hence was the most 
suitable candidate for the role of group leader. The group leader was responsible 
for guiding the choices of technology together with ensuring a schedule for the 
timely completion of the project. In addition, he was also responsible for 
managing the team during evaluations of the project requirements and the 
suitability of different technology stacks in fulfilling our design goals.
\\
\\
\noindent
More specifically, our team members were initially in charge of working on 
following areas.

\begin{itemize}
  \item Client-side presentation - Nuri
  \item Client-side interaction - Nuri / Han
  \item Server-side processing - Bradley / Piotr / Han
  \item Database set-up - Bradley / Piotr
\end{itemize}

\noindent However as some tasks take longer than others, all of us ended up 
working together of different parts of the project, outside the initial setup.
This was enabled by our close working conditions which meant that we could quickly ask 
the person who had implemented particular piece of functionality to explain the ideas 
or help fix bugs. We believe this was a good thing as it allowed us all to gain a general 
understanding of the bits we weren't assigned to, as well as a deep 
understanding of the bits we were. This helped us achieve our aims of getting a 
feel of the whole web application stack.   

\subsection{Implementation Choices}

Although we chose to use HTML, CSS, and JavaScript for the frontend (for 
compatibility reasons), we have considered a wide variety of web development 
frameworks, ranging from full blown MVC frameworks such as AngularJS, MeteorJS, 
and Django templates, to minimalistic libraries such as jQuery, Backbone.js, and 
HTML5 Boilerplate. As the team was relatively new to web development, we decided 
to use frameworks that more closely represent the fundamental DOM layout to 
enhance our understanding. This means that HTML5 Boilerplate is a good choice 
because it consolidates the best practices of using modern web technologies 
without adding extra layers of abstraction. In addition, we have chosen to use 
the Bootstrap framework to enhance the visual presentation of our web app. 
Lastly, we decided to separate our data models from the views by fetching data 
asynchronously as JSON objects from our backend using RESTful APIs. Using 
RESTful APIs allows for easy integration with mobile clients while using JSON as 
the data format allows for transfer of rich data structures like nested objects 
and arrays.
\begin{comment}
  see also, closure, scala, dart, go, c++
\end{comment}
\\
\\
\noindent
We have picked a statically typed language, Java, as the implementation language 
for our backend. Although it may increase the amount of time needed to write our 
App, we believe the extra type safety combined with an active developer 
community makes it a good trade-off over other dynamically typed languages such 
as Python, node.js, and PHP. As most of us are familiar with Java, it is also 
easier to apply sound software design patterns that we have learnt over the 
course of our studies, such as singleton pattern or dependency 
injection/inversion. With the support of testing frameworks like JUnit and 
Mockito, we are able to provide good test coverage on our backend code base to 
ensure long term maintainability. This is especially important for our target 
market - enterprise web apps (see our App description). More generally, Java 
code is more habitable and allows for easier extension with its wide range of 
third party libraries. We also believe it will keep our options open going 
forward as it is very portable and platform independent.
\\
\\
%might want to expand on the use of Tomcat and PostgreSQL
\noindent
Lastly, we have decided to use Apache Tomcat as the web container and PostgreSQL 
as the database engine because they have been conveniently set up by CSG on lab 
machines.

\subsection{Design Processes}

As we tried to create as little coupling as possible between the front end and 
the back end, we used slightly different processes in designing and creating 
each one.  
\\
\\
\noindent For the front end we decided to first build a basic UI without any 
functionality. It allowed us to get an interface that was easy to understand, 
usable and fluid. It meant that half of the group could just focus on that for 
the start of the project. Next we added some basic functionality to see if it 
worked well and adjusted things if we thought they could be improved. From there 
we performed iterations of adding UI features $\rightarrow$ adding in the 
functionality $\rightarrow$ adjusting to achieve the desired look and feel of 
the components.
\\
\\
\noindent The back end design started with sketching out an outline the for API 
which achieved the basic functionality that we needed for the first set of 
features. Then we spent a fair amount of time re-factoring to improve 
readability , trying to include as many best practices as we could find and 
writing some tests to ensure that we hit all of the edge cases. Then we joined 
the design process for the frontend in iterations of adding features 
$\rightarrow$ wiring them up to the front end $\rightarrow$ re-factoring if 
possible.     

\subsection{Backup Systems}

%Im not 100% sure what to write here
Gitlab came up as our method of choice for version control and backups (the 
choice was arbitrary). It has allowed us to ensure that all of our committed 
work is backed up and helps organise the project into different branches so that it was less 
likely for any work to be accidentally altered or deleted. As far as database backups, 
we have not needed one so far as we have been regularly wiping it for table 
changes or because of inconsistent data used for testing. 

\section{Program and Implementation}

The basics of our application have already been covered in introduction and 
project management sections. Now we will go into details, starting with the 
overall system architecture and then going into the specific parts of the 
system.

\subsection{Overview}

\noindent Our system contains three independent parts, the front end (static web 
pages + Javascript to provide calls to the API) the back end (the API, provides 
methods to access the main functionality of our system) and the database. The 
interactions are shown in the diagram below.

\begin{figure}[H]
\centerline{\includegraphics[scale=0.58,trim=0 0 100 0]{sysarch}}
\caption{System Architecture Diagram}
\end{figure}    


\subsection{Back End}

\subsubsection{Overview}
When designing the API of the back end, we followed REST architecture style and 
aimed to implement the the Level 2 of the Richardson Maturity Model. Our back 
end contains multiple servlets which allow the calenadr resources to be uniquely 
identified. We also used proper semantics of HTTP methods and returned various 
HTTP status codes to represent the effects of user's actions. JSON format was 
used to send client requests to the API. The object is then processed by the 
back end and a response object is returned (also serialized as a JSON string), 
the response will either be an error response, a success response (if no data 
was expected) or the appropriate data response object. This design could be used 
in the future to easily allow either clients to be written on different 
platforms or for people to build other tools to interact with the data and 
extend back end functionality. 
\\
\\
\noindent For the main bulk of the API, we have a range of servlets which 
respond to requests. They all follow a similar structure:
\begin{quote}
\centering
Get the request object $\rightarrow$ Check user can make the request 
$\rightarrow$ Call the related database interface method with the request object 
$\rightarrow$ Check for and deal with errors $\rightarrow$ Get/Create the 
correct response object $\rightarrow$ Set the response to be returned to the 
user.
\end{quote}
Both the serialization/deserialization and basic authentication (is the session 
ID valid and whose is it) are done though filters which all of the 
request/response objects get run though.
\\
\\
\subsubsection{User verification and session management}
\noindent We have implemented a complete login and session management system. We 
allow the users to create and account or login via calls to our API, on return 
we set a cookie on the users client with there session ID. All subsequent calls 
to the API must contain this session ID as to identify who the client is and 
give them the correct privileges. The session IDs expire after a certain amount 
of time to increase security. However, as to not inconvenience the user, if they 
make any request to the API with a session that is about to expire it gets 
automatically refreshed and a new session ID is sent to them. When the user logs 
out of the systems the session IDs are deleted so they can not be used to 
compromise the users identity.
\\
\\
\noindent In the login system we have been very careful how to deal with the 
users password, we opted to use the provided Java cryptographic library SecretKey class 
to perform the encoding of the password and salt. The salt is generated from the 
Java security library’s SecureRandom class, and is saved along with pass hash 
into the database. We have also set up a verification system so that we know the 
user of each account has the email they signed up with. It works by sending a 
code to the user's email after they register that they have to enter the first 
time they log in. As for this we had to add email support we also included a 
password reset function which sends a temporary password that the user can log 
on with.  
\\
\\
\subsubsection{Interfacing with the database}
\noindent We decided to create one class to wrap all of the database accesses, 
which lead to the simplification of the servlet classes. It also allowed us to 
group all of the related SQL queries onto the response classes, whose methods 
the database interface class methods would use in order to retrieve the relevant 
queries. When designing database interface class, we made and extensive use of 
\textbf{strategy pattern}. We generalized the database interactions into four 
methods: \verb|query, insert, update| and \verb|delete| that take as a parameter 
an interface, which represents an appropriate SQL query (\verb|SELECT| query, 
\verb|INSERT| query, \verb|UPDATE| query, \verb|DELETE| query), and hence allows 
the methods to be reused for different queries, provided that they implement 
appropriate interface. This made adding new features quick and easy as there 
were only a set of few places that needed to be changed: the query added to the 
relevant response object, a method added to the database interface (in some 
cases) and a servlet method created to deal with the request. In our 
implementation of response objects \textbf{template method pattern} was used, all 
objects are subclasses of Response class, which implements pieces of 
functionality common to all of them, which allowed us to limit the amount of 
code duplication. We also made a use of \textbf{singleton pattern} when creating 
a DataSourceProvider class, which initialises the database connection. The 
motivation for using the singleton pattern in this case is to ensure that no 
more than one data pool connection is created. 

\subsubsection{Chat Server}
Initially we were going to use a off the shelf, widely used XMPP server such as 
ejabberd. However, we realized it would not integrate well with our current 
back-end in various ways such as authentication and sending dynamic updates of 
JSON objects. Therefore, we implemented our websocket based chat server. In its 
core it has ChatMessage objects which are routed by the server. A ChatMessage 
has six fields which are inspired by the TCP protocol. The ChatMessage object is 
sent as JSON strings through the websocket connections. The fields are:
\begin{description}
\item[type] - Specifies what the payload contains. For example, "update/event" 
indicates that it is an update message containing an event object in the payload 
which is updated.
\item[destinationIds] - An array of user Ids which this message is addressed at.
\item[sourceId] - The user Id which created this chat message. If it is an 
update from the server, then the value is -1 indicating that the message 
originated at the server and not sent by an actual user.
\item[date] - A timestamp when the message was created, not sent. This is mainly 
used to store messages offline.
\item[storeOffline] - Boolean field to indicate whether message should be stored 
if the destination Id is currently offline.
\item[payload] - The actual data that is sent which is serialized to JSON as 
well.
\end{description}
\\
\\
\noindent Using this structure, all the Chat Server does is route messages based on their 
destination Ids and store in the database if they are offline. To know which 
sockets to route to, the server maintains a ConcurrentMap of UserIds to 
ConcurrentHashSet of open Sessions. Hence, even if the same user logs on 
different machines, the server is able to route the message to all of the open 
sessions of the user (which is not a standard feature of XMPP servers which 
implement sessions priority to route).

\subsubsection{Dynamic Updates}
In order to prevent the necessity to refresh the page every time calendar or 
event data changes, we had to design a mechanism which would allow the server to 
dynamically inform the user about the 
changes of the calendar or event data an re-render the parts of the view 
affected by the change instantly, making the website more responsive. The 
technology that we utilised to 
achieve our aim are WebSockets, which are a good fit due to the widespread 
browser support, responsiveness and the fact that they are 
already utilised by the chat server, hence
the client could re-use the same TCP connection to send also the updated data.
\\
\\
\noindent Our solution utilises \textbf{observer pattern}, by subscribing to a given 
calendar the user (observer) registers his intent to receive the relevant 
dynamic updates. Every time event is updated 
or added to/deleted from the calendar, all the users subscribed to the 
corresponding calendar, which are currently online, will receive an update. In 
order to make filtering the users
subscribed to a given calendar faster and reduce the database traffic, we used 
\textbf{caching} to store the mapping between calendar IDs and user IDs of 
subscribers. The cache was implemented
as a ConcurrentMap of Integers and Sets. To ensure thread-safety when accessing 
and modifying the set of subscribers of a given calendar, a \textbf{decorator 
pattern} was used and the standard
HashSet was wrapped inside a ConcurrentHashSet class, decorated with a 
read-write lock.

\subsection{Front End}
\subsubsection{Static files with AJAX}
The front-end structure is based on the HTML5 boiler plate 
(https://html5boilerplate.com/) organizing our static files under directories 
such as js, css, fonts. The crucial aspect of the front end is that it only 
contains static files under the root directory, in other words, there isn't any 
data in the html files or in the JavaScript code. Our app only contains two main 
pages: the login page and the main calendar app page. When either of these files 
are requested, they initially bind all the UI functions such as button presses 
and then make AJAX calls to our API. This asynchronous data loading has several 
implications on the how the app front-end behaves. Firstly, the front end acts 
only as a view in the overall model-view-controller structure of the app. Hence, 
it is completely independent of the data source, i.e. the back end. This 
separation allows us to host the static files (i.e. the view etc) somewhere else 
such as Google App Engine where serving static files are free, and only host the 
API. Secondly, front end can load without any data. Even though there will be an 
empty frame saying that the user, for example, does not have any events, it will 
be loaded nonetheless. This AJAX loading method ensures that the front-end is 
visible to the user even if the back-end encounters a fatal error.

\subsubsection{Data to Rendering}
The JavaScript code is organized into several layers that separate the concerns 
of individual functions and parts. Firstly, the code is divided into three sections: 
user, calendar and event code. Each section is refactored into a separate 
JavaScript file which deal with getting the data to rendering it on the screen. 
All of the data corresponding to the state of the app is stored in a global 
JavaScript object called app. This object is global because different parts of 
the application such events, calendars functions require the data of other 
sections of the app rather than making a train wreck of passed parameters to the 
function call hierarchy. They all follow the similar pattern of:
\begin{description}
\item[Refresh] - Initially the refresh function is called for the corresponding 
part. For example, refreshCalendars updates user's calendars. This function is 
responsible for \textbf{fetching} the data from the back end using the provided 
API calls. Then it stores the information under the global app object, 
app.calendars in this case. Finally, the rendering is invoked.
\item[Render Items] - This function first clears the whatever is currently 
rendered and then for each of the item in the corresponding section of the 
global app object calls the appropriate render item function.
\item[Render Item] - The heart of the rendering happens in functions which 
render individual items such as events on the screen. The rendering is achieved 
by created new DOM elements using jQuery which are then inserted in the right 
place on the screen. For example, renderEvent function renders a single event on 
the main calendar table.
\end{description}
It is important to note that this refresh to rendering cycle only happens when 
the page is loaded for the first time. Then only the rendering functions are 
used in correspondence with the dynamic update messages received from the 
back end. This method significantly reduces the AJAX overhead of fetching the 
entire data again even if only one event is changed.

\subsubsection{User Interactions}
The basic idea is that there are two types of users (per calendar), the 
administrators and the ordinary users (subscribers). They will interact with the 
calendars in different ways and have various typical uses. Note that any user can 
create a calendar or join other calendars so these roles are not absolute.
\\
\\
\noindent
A subscriber of a calendar is able to see calendar’s events in a clear manner 
and easily access the details of an event, like time, location, duration and 
short description, together with a feedback on the interest about the event 
(number of people already registered to the event). Users also need the ability 
to sign up for the event they wish to attend (depending on the event this may 
require some requirements being met). In addition, users can change between the 
calendars they are subscribed to and create a dynamically-customisable view 
including only the calendars that interest them at a particular time. The view 
enables the users to preview the information about events occurring in a certain 
time interval (for example, a week from a selected). Finally, all users of the 
application are be able to create calendars and gain administrator privilege on 
them.
\\
\\
\noindent
The administrations of a calendar have extra abilities in addition to those of a subscriber. They can create, modify details of and delete any event on their calendars along with the ability to see the number of people who have signed up and their identities. In addition, they can delete whole calendars and invite people to subscribe via an invite code which they can give out to anyone who may be interested.
\\
\\
\noindent
We also decided to implement a chat feature whereby subscribers to the admins and editors of calendars and admins/editors and chat among themselves. It is intended to be used to aid the event organisation process as well as to allow people to pose questions about the events to the organisers. This give our app some much needed user-user interaction to hopefully extend the time that users will spent interacting with the app and the other people who use it.
\\
\\
\noindent Some examples of typical user interaction can be found in \cref{sec:fd} (as in the milestone report).   

\subsubsection{UI Design}
TODO: Put a screen shot, discuss why the elements are where they are. Talk about 
the flow of the creating events to changing / joining stuff. Mention

\subsubsection{Mobile Optimizations}
As mobile traffic has been the main driving force of internet growth today, we wanted to create a great experience for our users on mobile devices. The bootstrap framework includes a responsive grid layout system that automatically adapts to different screen sizes, however it was not enough to fulfill the needs of our application. In our application, we have handcrafted a unique user interface based on iterative feedback from weekly meetings with actual users of our application and in order to bring the same optimized experience to mobile, we had to fine tune the media queries for phone, tablet, and desktop by ourselves.

As seen from the screenshots in UI Design, the transition of our user interface across different form factors is seamless. The sliding in and sliding out animations of the application drawers were rendered completely in CSS to achieve the maximum possible performance on less powerful mobile devices. Days on the calendar that do not fit into the the current viewport size are also hidden by CSS styles rather than being removed from the DOM for the same performance reasons. To enable accurate detection of swipe gestures, we customized a subset of jQuery Mobile UI component and added custom event handlers to “swipeleft” and “swiperight”. Last but not least, the default navigation bar provided by Bootstrap has to be customized to remove any collapsable toggles on mobile screens so that it does not interfere with our custom UI.

\section{Libraries}
Libraries that are used in the front-end:
\begin{description}
\item[jQuery] - We use version 2.1.4 which like any other version is licensed 
under the MIT license that permits any applicable usage. Although we could, we 
did not modify any sections of the library. https://jquery.com
\item[jQuery Datetime Picker] - We use this library to bind picking of date time 
input fields in the forms. Like jQuery the library is under the MIT license. 
http://xdsoft.net/jqplugins/datetimepicker/
\item[Bootstrap] - Bootstrap is also licensed under MIT. 
http://getbootstrap.com/
\item[Toastr Js] - Used to conveniently display toast messages.
MIT license. https://github.com/CodeSeven/toastr
\item[Javascript Timezone] - Used for automatic timezone detection on the client.
MIT license. http://pellepim.bitbucket.org/jstz/
\item[Chat JS] - Facebook sytle chat UI. MIT license. http://chatjs.net/
\end{description}
It is important to note that all of the libraries the front-end uses are under 
the MIT license which means that we have the permission to reuse within 
proprietary software, for example we do not disclose the source code and charge 
for the usage, provided that all copies of the licensed software include a copy 
of the MIT License terms and the copyright notice. \\ \\
Libraries that are distributed in the back-end:
\begin{description}
\item[Google GSON] - The GSON library is used to serialize and deserialize Java 
objects into Json strings when they are sent to and recieved from the front-end. 
It is licensed under the Apache License 2.0 which means we are allowed to redistribute
the software and our derivative work as long as we include the copyright notice.
https://code.google.com/p/google-gson/
\item[Google Guava] - Implementations of convenient data structures like immutable 
collections, caches, preconditions, etc. Apache License 2.0. https://github.com/google/guava
\item[Apache Commons Validator] - Well tested validation methods for date strings, 
alphanumeric input, etc. Apache License 2.0. https://commons.apache.org/
\item[JavaMail API] - API for sending emails from our backend. There is no
copyright to APIs so we can use it freely.
\item[Servlet API 3.0] - Annotations for conveniently creating servlets that
serves our REST backend. No copyright to APIs.
\item[PostgreSQL JDBC Driver] - The Java back-end is linked to the group 
database using the driver. Licensed under BSD license. 
https://jdbc.postgresql.org/
\end{description}

\section{Conclusion}
 \subsection{Piotr's Conclusion}
 Overall I am very satisfied with the outcome of our work. We managed to design 
a practical application which has a high chance of being used on Imperial 
Campus. Close cooperation with Imperial Outreach Office allowed us to gather 
many important insights that we used to enhance usability and add more features 
to our app. When it comes to my experiences with working on the project, I 
learned a lot about back end development, which was my primary focus, as I had 
never worked on the Internet application back end before and wanted to gain some 
skills in that area. I am very satisfied with my cooperation with the rest of 
the team, the atmosphere was friendly and due to daily meetings in the labs, we 
managed to develop the project at fast pace. On the downside, I did not manage 
to create a complete test suite for the implemented functionality, which should 
be done before the release of our app. However, due to the dynamically-changing 
requirements for our project (set by Outreach Office) the implementation had 
been changing a lot in the process, which made writing test suite difficult. 
 
 \subsection{Bradley's Conclusion}
 I think this project went fairly well, we managed to achieve most of the aims 
that we set out to do. I managed to get a good understanding of the whole web 
application building process. Although most of my work was on the database 
set-up and interaction I feel that it has also given me a decent understanding 
of the client facing aspects as well. If we were to redo this project I would 
have allocated more time to fully design the different sections of the system 
and how the interact before starting the implementation. We had to rework our 
back end design a few times in the early stages so that it would be primed for 
extension for the features that were considering to add in the future. Hopefully 
we have managed to structure the system in such a way that it should be 
relatively painless to continue to develop this application and add new features 
when we need to.  

 \subsection{Nuri's Conclusion}
 As far as the group projects go, I think this was best in many ways. Firstly, 
we were building an actual product with potential customers. It did not feel 
like after we click the yellow hand-in button on cate that it will become 
history. This aspect not only motivated the team but also pushed it to design, 
develop and possibly deploy right code. Secondly, building a web app I think 
incorporated many different technologies and allowed for a good work divide 
within the group. Anything from the front-end javascript code to database 
queries, there was a challenge for us which we solved using our combined 
knowledge and often along way argued which solution is considered best practice. 
Finally, having a human centred design perspective as we regularly met with 
people from the Outreach Office, added an extra business side to the project. We 
tried to implement their needs, but sometimes even discarded them as we had to 
carefully judge what could be done within the time limit we got which is quite 
similar to a real world software development cycle. Overall, I think as a team, 
we went beyond the basic requirements of the coursework and built a viable web 
application recreates the event organization experience on campus.

\subsection{Han's Conclusion}
This webapp project has been a great learning experience for me. It is the first time I’m building an enterprise application and that has impacted our design and implementation decisions significantly. For instance, testing has been a priority from the start of the project because the application has to be stable for enterprise users from day one. It was also my first time building a web application with the first mobile approach. This brought a new perspective to the way I think about CSS styles, starting from a mobile screen size and gradually expanding to bigger screen sizes with media queries. Overall, the team has accomplished what we set out to do, to make a web application that replaces the current email system used by school administrators to organise events.

\appendix
\label{appendix}

\section{User Interaction Diagrams}
\label{sec:fd}
\begin{figure}[H]
\centering
\begin{minipage}{.4\textwidth}
  \centering
  \includegraphics[width=1\linewidth]{user_event}
  \caption{Use flow of a user signing up}
\end{minipage}%
\begin{minipage}{.7\textwidth}
  \centering
  \includegraphics[width=0.9\linewidth]{calendar_user}
  \caption{Use flow of a new user creating a calendar}
\end{minipage}
\end{figure}

\end{document}
